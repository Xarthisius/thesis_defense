\documentclass{beamer}

\mode<presentation>
{
%  \usetheme{Boadilla}
%  \usetheme[blue,compress,numbers,nonav]{Trondheim}
   \usetheme[blue,compress]{Madrid}
  \usefonttheme[onlymath]{serif}
  \setbeamercovered{transparent}
}

\usepackage{tikz}
\usetikzlibrary{arrows}
\usepackage{multimedia}
\usepackage[buttonsize=10.0pt]{animate}
\usepackage[normalem]{ulem}
\usepackage{hyperref}

\usepackage[polish]{babel}
\usepackage[utf8]{inputenc}
\usepackage{mathptmx}
\usepackage{helvet}
\usepackage{courier}
\usepackage{graphics}
\usepackage[T1]{fontenc}
\usepackage[absolute,overlay]{textpos}
\setlength{\TPHorizModule}{1mm}
\setlength{\TPVertModule}{1mm}
\def\strike#1#2{\temporal<#1>{#2}{\sout{#2}}{\sout{#2}}}
\newcommand\earth{\oplus}%
\def\Mearth{\hbox{$\mathrm{\thinspace M_{\earth}}$}}
\def\Mceres{\hbox{$\mathrm{\thinspace M_{\textrm{ceres}}}$}}
\def\K{\hbox{$\thinspace\textrm{K}$}}
\def\Msun{\hbox{$\mathrm{\thinspace M_{\odot}}$}}
\def\AU{\hbox{$\thinspace\textrm{AU}$}}
\def\yr{\hbox{$\thinspace\textrm{lat}$}}

\selectlanguage{polish}

\title[Powstawanie planetozymali]
{Powstawanie planet w wyniku połączonego działania niestabilności płynowych w dyskach protoplanetarnych}
\author[Kacper Kowalik{}]{Kacper Kowalik}
\institute[CA UMK]{Centrum Astronomii UMK}
\date[15.12.2014]{Obrona rozprawy doktorskiej}

\subject{planet formation}

\begin{document}

\begin{frame}
  \titlepage
\end{frame}

\begin{frame}
\frametitle{Paradygmat}
Planety formują się w dyskach poprzez zderzenia i zlepianie się cząstek (\alert{hipoteza planetozymali} Safronov, 1969).
\begin{itemize}
\item Od pyłu do planetozymali:
\begin{description}
\uncover<1>{\item[$\mu$m$\rightarrow$cm:] Oddziaływania międzycząsteczkowe (van der Waals, wiązania wodorowe) prowadzą do koagulacji (JD).}
\only<1-2>{\item[cm$\rightarrow$km:] Połączone działanie niestabilności płynowych (MRI, KHI, SI, GI, \ldots).}
\end{description}
\uncover<1>{
\item Od planetozymali do protoplanet:
\begin{description}
\item[km$\rightarrow 10^3$km:] Grawitacja.
\end{description}
\item Od protoplanet do planet:
\begin{description}
\item[Gazowe olbrzymy:] Akrecja gazowych otoczek.
\item[Planety skaliste:] Zderzenia protoplanet.
\end{description}}
\end{itemize}
\end{frame}

\begin{frame}

\pgfdeclareimage[interpolate=true,height=0.4\textheight]{d1}{dullemond1}
\pgfdeclareimage[interpolate=true,height=0.4\textheight]{d2}{dullemond2}
\pgfdeclareimage[interpolate=true,height=0.35\textheight]{mri}{mri}
\pgfdeclareimage[interpolate=true,height=0.36\textheight]{khi}{khi}

%\frametitle{Bariera 1m @ 1 AU}
\begin{block}{Procesy hamujące wzrost cząstek pyłu}
\begin{itemize}
\item Duża względna prędkość cząstek prowadzi do fragmentacji przy zderzeniu (Blum \& Wurm 2008)
\uncover<2->{\item Silne ograniczenie skali czasowej na skutek radialnego dryfu pyłu.}
\uncover<3->{\item Turbulencja (MRI) przeciwdziała sedymentacji.}
\uncover<4->{\item Sedymentacja prowadzi do niestabilności KH.}
\end{itemize}
\end{block}
\begin{center}
\only<1>{\pgfuseimage{d1}}
\only<2>{\pgfuseimage{d2}}
\only<3>{\pgfuseimage{mri} \\
   {\scriptsize Johansen A., \& Klahr H., \emph{ApJ}, \bf{634}}
}
\only<4>{\pgfuseimage{khi} \\
   {\scriptsize Johansen A., Henning Th., \& Klahr H. \emph{ApJ}, \bf{643}}
}  
\end{center}
\end{frame}

\begin{frame}
\frametitle{Przechwytywanie pyłu przez maskima w rozkładzie ciśnienia gazu}
\pgfdeclareimage[interpolate=true,height=0.8\textheight]{d3}{dullemond3}
\pgfdeclareimage[interpolate=true,width=0.45\textwidth]{a1}{arm1}
\pgfdeclareimage[interpolate=true,width=0.45\textwidth]{a2}{arm2}

\begin{center}
\only<1>{
  \pgfuseimage{a1} \pgfuseimage{a2} \\
  {\scriptsize Rice, W.~K.~M., Lodato,~G., Pringle,~J.~E., Armitage,~P.~J., Bonnell,~I.~A., (2004), \emph{MNRAS}, \bf{355}}
}
\only<2>{
  \pgfuseimage{d3}
}
\end{center}
\end{frame}

\begin{frame}
   \frametitle{Niestabilność strumieniowa - składniki}
   \pgfdeclareimage[interpolate=true,width=0.78\textwidth]{si}{streaming_jacquet}
   \begin{center}
      \pgfuseimage{si}\\
      {\scriptsize Jacquet E., Balbus S., Latter H. (2011) \emph{MNRAS}, \bf{415}}
   \end{center}
   \begin{itemize}
      \item Siła Coriolisa równoważy gradient ciśnienia gazu -- \emph{równowaga geostroficzna} $(\Omega)$
      \item Globalny gradient ciśnienia w dysku zapewnia ciągły, radialny dryf pyłu $(\eta)$
      \item $\Omega, \eta$ -- \alert{w lokalnym przybliżeniu zapewniają nieskończone źródło energii.}
   \end{itemize}
\end{frame}

\begin{frame}
\frametitle{Niestabilność strumieniowa - historia}
\pgfdeclareimage[interpolate=true,width=.48\textwidth]{aj}{stream_aj}
\begin{columns}
   \begin{column}{0.5\textwidth}
      {\scriptsize
      \begin{enumerate}
         \item Youdin \& Goodman (2005) odkyrywają (na nowo) proces, w którym cząsteczki pyłu na skutek tarcia aerodynamicznego
	 	mogą tworzyć przejściowe zagęszczenia.
         \item Johansen \& Youdin (2007) pokazują, że zagęszczenia pyłu spowodowane niestabilnością strumieniową, w warunkach panujacych
	       w dyskach protoplanetarnych, mogą osiągąć gęstości o trzy rzędy wielkości większe niż początkowa gęstość pyłu.
	 \item Pozwala to zapostulować Johansenowi i in. (2007), że niestabilność strumieniowa może prowadzić do wytworzenia się w dysku
	       protoplanetarnym grawitacyjnie związanych obiektów, utworzonych z ziaren pyłu $\ll1m$.
      \end{enumerate}
      }
   \end{column}
   \begin{column}{0.5\textwidth}
      \begin{center}
         \pgfuseimage{aj}\\
         {\scriptsize Johansen A., \& Youdin A. (2007), \emph{ApJ}, \bf{662}}
      \end{center}
   \end{column}
\end{columns}
\end{frame}

\begin{frame}
   \frametitle{,,Czy to dalej działa w modelu globalnego dysku?''}
   \begin{itemize}[<+->]
      \item \uncover<1>{Model ,,wszystkomający'' vs} analiza uproszczonego problemu.
      \item Musimy wykluczyć szereg zjawisk mogących przeciwdziałać, bądź udawać niestabilność strumieniową, 
         takich jak: \strike{3}{MRI}, \strike{4}{niestabilność barokliniczna}, \strike{5}{KHI}.
      \only<3->{\begin{itemize}}
         \only<3->{\item czysta hydrodynamika}
         \only<4->{\item izotermiczy gaz}
         \only<5->{\item brak pionowej składowej w zewnętrznym potencjale grawitacyjnym}
      \only<3->{\end{itemize}}
   \end{itemize}
\end{frame}

\begin{frame}
%\frametitle{Current state of PIERNIK art}
\only<1>{
\pgfdeclareimage[interpolate=true,width=0.7\textwidth]{logoHD}{logoHD}
\pgfdeclareimage[interpolate=true,height=2.5cm]{arfur}{arfur}
\pgfdeclareimage[interpolate=true,height=2.5cm]{wolt}{wolt}
\pgfdeclareimage[interpolate=true,height=2.5cm]{szef}{szef}
\pgfdeclareimage[interpolate=true,height=2.5cm]{jd}{jd}
\begin{center}
\pgfuseimage{logoHD}
\end{center}
\begin{center}
\begin{tabular}{cccc}
   \pgfuseimage{jd} & 
   \pgfuseimage{arfur} & 
   \pgfuseimage{szef} &
   \pgfuseimage{wolt} \\ 
   \scriptsize{Joanna Dr\k{a}\.zkowska} &
   \scriptsize{Artur Gawryszczak} & 
   \scriptsize{Micha\l{} Hanasz} &
   \scriptsize{Dominik W\'olta\'nski}
\end{tabular}
\href{http://piernik.astri.umk.pl/}{http://piernik.astri.umk.pl/}
\end{center}
}
\only<2>{
\begin{itemize}
   \item Solver hydrodynamiczny -- Relaxing TVD (Jin \& Xin, 1995; Trac \& Pen, 2003)
   \item Geometria cylindryczna (Skinner \& Ostriker, 2010), wielopłynowość, oddziaływanie między pyłem, a gazem przy użyciu ,,pół-niejawnego'' schematu (Tilley \& Balsara, 2008).
   \item Samograwitacja (multigrid, multipole, direct FFT)
   \item Efektywne zrównoleglenie zagadnienia.
\end{itemize}
Warunki początkowe:
{\scriptsize
\begin{enumerate}
   \item MMSN: $\Sigma(R) = 1700\cdot r^{-1.5}$ g cm$^2$, $T(R_0) = 150 ( R_0 / 1 \textrm{AU} )^{-0.429}$ K,
   \item 2D: $[2\div7,0\div2\pi,-0.146\div0.146]$ w $N=(5120, 1, 300)\div(20480, 1, 1200)$,
   \item 3D: $[2\div7,0\div\pi/6,-0.1875\div0.1875]$ w $N=(2560, 512, 192)$,
   \item zaburzenie prędkości pyłu $\sim10^{-6} e^{-i(k_x x + k_z z)}$ albo biały szum,
   \item ,,dumping zones'' $\frac{\textrm{d}X}{\textrm{d}t} = -\frac{X-X_0}{\tau}f(r)$,
   \item Czas numerycznej relaksacji układu $T_{\textrm{relax}} = 10$ yr.
\end{enumerate}
}
}
\end{frame}

\begin{frame}
\pgfdeclareimage[interpolate=true,width=\textwidth]{2d1}{2d_1}
\pgfdeclareimage[interpolate=true,width=\textwidth]{2d2}{2d_2}
\pgfdeclareimage[interpolate=true,width=\textwidth]{2d3}{2d_3}
\begin{center}
   \only<1>{\pgfuseimage{2d1}}
   \only<2>{\pgfuseimage{2d2}}
   \only<3>{\pgfuseimage{2d3}}
\end{center}
\end{frame}

\begin{frame}
\frametitle{Niestabilność strumieniowa 2D (BA)}
\pgfdeclareimage[interpolate=true,width=.5\textwidth]{ba1}{BA1}
\pgfdeclareimage[interpolate=true,width=.5\textwidth]{ba2}{BA2}
\pgfdeclareimage[interpolate=true,width=.5\textwidth]{ba3}{BA3}
\pgfdeclareimage[interpolate=true,width=.5\textwidth]{ba4}{BA4}
\pgfdeclareimage[interpolate=true,width=.45\textwidth]{ba1kk}{BA1_kk}
\pgfdeclareimage[interpolate=true,width=.45\textwidth]{ba2kk}{BA2_kk}
\pgfdeclareimage[interpolate=true,width=.45\textwidth]{ba3kk}{BA3_kk}
\pgfdeclareimage[interpolate=true,width=.45\textwidth]{ba4kk}{BA4_kk}
\begin{columns}
   \begin{column}{0.54\textwidth}
      \begin{center}
      	 \only<1>{\pgfuseimage{ba1}}
      	 \only<2>{\pgfuseimage{ba2}}
      	 \only<3>{\pgfuseimage{ba3}}
      	 \only<4>{\pgfuseimage{ba4}}
      \end{center}
   \end{column}
   \begin{column}{0.46\textwidth}
      \begin{center}
         \only<1>{\pgfuseimage{ba1kk}}
         \only<2>{\pgfuseimage{ba2kk}}
         \only<3>{\pgfuseimage{ba3kk}}
         \only<4>{\pgfuseimage{ba4kk}}
      \end{center}
   \end{column}
\end{columns}
\end{frame}

\begin{frame}
\frametitle{Niestabilność strumieniowa 2D (AB)}
\pgfdeclareimage[interpolate=true,width=.5\textwidth]{ab1}{AB1}
\pgfdeclareimage[interpolate=true,width=.5\textwidth]{ab2}{AB2}
\pgfdeclareimage[interpolate=true,width=.45\textwidth]{ab1kk}{AB1_kk}
\pgfdeclareimage[interpolate=true,width=.45\textwidth]{ab2kk}{AB2_kk}
\begin{columns}
   \begin{column}{0.54\textwidth}
      \begin{center}
      	 \only<1>{\pgfuseimage{ab1}}
      	 \only<2>{\pgfuseimage{ab2}}
      \end{center}
   \end{column}
   \begin{column}{0.46\textwidth}
      \begin{center}
         \only<1>{\pgfuseimage{ab1kk}}
         \only<2>{\pgfuseimage{ab2kk}}
      \end{center}
   \end{column}
\end{columns}
\end{frame}

\begin{frame}
\frametitle{Liniowa analiza stabilności}
\only<1>{
Początkowy etap rozwoju niestabilności strumieniowej (faza liniowa) daje się ,,potraktować'' analitycznie w układzie ,,kostki ścinanej'', poprzez linearyzcję 8 równań PDE, opisujących 8 wielkości fizycznych. Otrzymujemy informację na temat stanu układu zapisaną jako superpozycję fal płaskich
\[ \tilde{f}\exp\left[i\left(k_x x + k_z z- \omega t\right)\right],\textrm{ gdzie }
   \tilde{f}=\left[\tilde{\rho_d},\tilde{w}_x,\tilde{w}_y,\tilde{w}_z,\tilde{\rho_g},\tilde{u}_x,\tilde{u}_y,\tilde{u}_z\right]. \]
\[ \mathbf{A(\omega)} \tilde{f} = \vec{b} \] 
Schemat analizy:
\begin{itemize}
\item Wybieramy z naszej globalnej symulacji obszar na tyle mały, że możemy go traktować jako lokalne przybliżenie dysku (,,łatka'').
\item Dla zbioru stałych określającego naszą ,,łatkę'' znajdujemy rozwiązania zespolone dla równania 8 stopnia ($\det|\mathbf{A(\omega)}|=0 \Rightarrow \omega$).
\item Wyznaczamy wzajemne relacje pomiędzy amplitudami dla ustalonego jednego zaburzenia np. $\rho_d$. ($\tilde{f} = \mathbf{A}^{-1}\vec{b}$) . 
\end{itemize}
}

\only<2>{
\pgfdeclareimage[interpolate=true,width=0.85\textwidth]{a1}{amps}
\begin{center}\pgfuseimage{a1}\end{center}
}
\only<3>{
\pgfdeclareimage[interpolate=true,width=0.75\textwidth]{a1}{fale}
\begin{center}\pgfuseimage{a1}\end{center}
}
\only<4>{
\pgfdeclareimage[interpolate=true,width=0.5\textwidth]{fig9a}{fig9a}
\pgfdeclareimage[interpolate=true,width=0.5\textwidth]{fig9b}{fig9b}

\begin{columns}
   \begin{column}{0.5\textwidth}
      \begin{center}
         \pgfuseimage{fig9a}
      \end{center}
   \end{column}
   \begin{column}{0.5\textwidth}
      \begin{center}
         \pgfuseimage{fig9b}
      \end{center}
   \end{column}
\end{columns}
}
\end{frame}

\begin{frame}
\frametitle{Pełne symulacje 3D} 
\pgfdeclareimage[interpolate=true,width=0.5\textwidth]{pl1}{slice_nosg_04.png}
\pgfdeclareimage[interpolate=true,width=0.5\textwidth]{pl2}{slice_sg_04.png}
\pgfdeclareimage[interpolate=true,width=0.95\textwidth]{proj}{proj_sg}

\only<1>{
\begin{columns}
   \begin{column}{0.5\textwidth}
      \begin{center}
         \pgfuseimage{pl1}
      \end{center}
   \end{column}
   \begin{column}{0.5\textwidth}
      \begin{center}
         \pgfuseimage{pl2}
      \end{center}
   \end{column}
\end{columns}
}
\only<2>{
   \begin{center}
      \pgfuseimage{proj}
   \end{center}
}
\only<3>{
\pgfdeclareimage[interpolate=true,width=0.5\textwidth]{hnosg}{hist2d_nosg_01.png}
\pgfdeclareimage[interpolate=true,width=0.5\textwidth]{hsg}{hist2d_sg_01.png}
\begin{columns}
   \begin{column}{0.5\textwidth}
      \begin{center}
         \pgfuseimage{hnosg}
      \end{center}
   \end{column}
   \begin{column}{0.5\textwidth}
      \begin{center}
         \pgfuseimage{hsg}
      \end{center}
   \end{column}
\end{columns}
}
\only<4>{
\pgfdeclareimage[interpolate=true,width=0.5\textwidth]{hnosg}{hist2d_nosg_02.png}
\pgfdeclareimage[interpolate=true,width=0.5\textwidth]{hsg}{hist2d_sg_02.png}
\begin{columns}
   \begin{column}{0.5\textwidth}
      \begin{center}
         \pgfuseimage{hnosg}
      \end{center}
   \end{column}
   \begin{column}{0.5\textwidth}
      \begin{center}
         \pgfuseimage{hsg}
      \end{center}
   \end{column}
\end{columns}
}
\only<5>{
\pgfdeclareimage[interpolate=true,width=0.5\textwidth]{hnosg}{hist2d_nosg_03.png}
\pgfdeclareimage[interpolate=true,width=0.5\textwidth]{hsg}{hist2d_sg_03.png}
\begin{columns}
   \begin{column}{0.5\textwidth}
      \begin{center}
         \pgfuseimage{hnosg}
      \end{center}
   \end{column}
   \begin{column}{0.5\textwidth}
      \begin{center}
         \pgfuseimage{hsg}
      \end{center}
   \end{column}
\end{columns}
}
\only<6>{
\pgfdeclareimage[interpolate=true,width=0.5\textwidth]{hnosg}{hist2d_nosg_04.png}
\pgfdeclareimage[interpolate=true,width=0.5\textwidth]{hsg}{hist2d_sg_04.png}
\begin{columns}
   \begin{column}{0.5\textwidth}
      \begin{center}
         \pgfuseimage{hnosg}
      \end{center}
   \end{column}
   \begin{column}{0.5\textwidth}
      \begin{center}
         \pgfuseimage{hsg}
      \end{center}
   \end{column}
\end{columns}
}
\only<7>{
\pgfdeclareimage[interpolate=true,width=0.5\textwidth]{hnosg}{hist2d_nosg_05.png}
\pgfdeclareimage[interpolate=true,width=0.5\textwidth]{hsg}{hist2d_sg_05.png}
\begin{columns}
   \begin{column}{0.5\textwidth}
      \begin{center}
         \pgfuseimage{hnosg}
      \end{center}
   \end{column}
   \begin{column}{0.5\textwidth}
      \begin{center}
         \pgfuseimage{hsg}
      \end{center}
   \end{column}
\end{columns}
}
\end{frame}

\begin{frame}
\frametitle{Czy powstają obiekty związane grawitacyjnie?}
\only<1>{
\begin{enumerate}
\item Identyfikacja wszystkich obiektów powiązanych topologicznie (\texttt{yt})
\item $\tilde{E}_{\textrm{kin}} = \
   \sum\limits_{i=1}^N \frac{m_i\tilde{\mathbf{v}}_i^2}{2}
   + \sum\limits_{i=1}^N \frac{m_i \left(\alpha c_s\right)^2}{2}$,
\item $\tilde{\Phi}_i(R,\varphi,z) = \Phi_i(R,\varphi,z) - \left<\Phi_i\right>_{z\varphi}(R).$
\end{enumerate}
\begin{block}{Kryterium związania grawitacyjnego}
\begin{equation}
\tilde{E}_{\textrm{kin}} < \sum\limits_{i=1}^N m_i\tilde{\Phi}_i \equiv E_{\textrm{pot}}
\end{equation}
\end{block}
}

\only<2>{
\pgfdeclareimage[interpolate=true,width=0.5\textwidth]{hnosg}{bmass_vs_time}
\pgfdeclareimage[interpolate=true,width=0.5\textwidth]{hsg}{nclumps_vs_time}
\begin{columns}
   \begin{column}{0.5\textwidth}
      \begin{center}
         \pgfuseimage{hnosg}
      \end{center}
   \end{column}
   \begin{column}{0.5\textwidth}
      \begin{center}
         \pgfuseimage{hsg}
      \end{center}
   \end{column}
\end{columns}
    {\scriptsize Lewy panel przedstawia ilość masy pyłu, która została zgromadzona w~grawitacyjnie związanych
     obiektach w~funkcji czasu. Wartość maksymalna osiągnięta po około 300
     latach (od $4$ do $5\Mearth$ dla $\alpha = 0.05$) stanowi około $13$ -- $14\%$
     całkowitej masy pyłu. Prawy panel przedstawia ilość grawitacyjnie
     związanych obiektów w~funkcji czasu i parametru $\alpha$.}
}
\only<3>{
   \pgfdeclareimage[interpolate=true,width=0.9\textwidth]{mh}{mass_hists}
   \begin{center} \pgfuseimage{mh} \end{center}
   {\scriptsize Histogramy masy obiektów związanych dla $\alpha = 0.01, 0.05, 0.1$
   (odpowiednio panel lewy, środkowy i prawy) w~zakresie $[0,600]\Mceres$.
   Szerokość pojedynczego przedziału wynosi $20\Mceres$. Na wewnętrznych
   panelach znajdują się histogramy dla obiektów o masach mniejszych niż
   $80\Mceres$. Szerokość pojedynczego przedziału na wewnętrznym panelu wynosi
   $4\Mceres$.}
}

\only<4>{
   \pgfdeclareimage[interpolate=true,width=0.9\textwidth]{mh}{mass_func}
   \begin{center} \pgfuseimage{mh} \end{center}
   {\scriptsize Złożenie histogramów masy obiektów związanych grawitacyjnie dla
   $\alpha = 0.05$ dla $T \in [250, 300]$ lat (czerwone punkty). Niebieska linia
   stanowi dopasowanie funkcji wykładniczej do liczby obiektów w~funkcji masy na
   przedziale $[40,200]\Mceres$}
}
\end{frame}

\begin{frame}
\frametitle{Podsumowanie}
\scriptsize
\begin{itemize}
   \item zarówno dwu- jak i trójwymiarowe symulacje quasi--globalnego dysku
      gazowo--py\-ło\-we\-go, w którym oba wzajemnie ze sobą oddziałujące poprzez siłę
      tarcia składniki są traktowane w przybliżeniu płynowym, stanowią wiarygodne
      narzędzie do badania niestabilności strumieniowej;
   \item algorytmy numeryczne zaimplementowane w kodzie \textsc{Piernik}
      pozwalają na odtworzenie z wystarczającą dokładnością liniowej fazy
      wzrostu niestabilności strumieniowej;
   \item niestabilność strumieniowa prowadzi do wytworzenia się w gazowo--pyłowym
      dysku obszarów, w których gęstość pyłu jest ponad stukrotnie wyższa niż
      maksymalna początkowa gęstość pyłu;
   \item niestabilność strumieniowa z uwzględnieniem samograwitacji materii
      prowadzi do wytworzenia się populacji związanych grawitacyjnie, pyłowych
      obiektów o spektralnym rozkładzie masy danym funkcją potęgową o wykładniku
      $-1.25\pm0.12$;
   \item masa pyłu zgromadzona w pojedynczych, grawitacyjnie związanych obiektach
      odpowiada ciałom o rozmiarach rzędu kilkudziesięciu do kilkuset
      kilometrów;
   \item całkowita masa pyłu zgromadzona w grawitacyjnie związanych obiektach w
      przedstawionym modelu 
      jest wystarczająca do wytworzenia pojedynczego jądra gazowego
      olbrzyma, bądź dwóch lub trzech planet skalistych;
   \item dalsze prace nad modelem są potrzebne w celu uściślenia ilościowych
      szacunków całkowitej masy, liczby i spektrum masowego powstających
      planetozymali.
\end{itemize}

\end{frame}

\begin{frame}
\end{frame}

\begin{frame}
\frametitle{Uwagi}
   \pgfdeclareimage[interpolate=true,width=0.5\textwidth]{mh}{chap1_sed}
\begin{columns}
   \begin{column}{0.5\textwidth}
      \begin{center}
      \pgfuseimage{mh} 
      \end{center}
   \end{column}
   \begin{column}{0.5\textwidth}
   \only<1>{
     {\scriptsize Rozkład widma energii dla gwiazdy HD 34282. Symbole oznaczają
      obserwacje różnymi metodami dla odpowiednich długości fali. Do obserwacji
      dopasowano następujące modele: linia przerywana model widma gwiazdowego
      dla obiektu typu A3V $(T\sim 8600\K)$, cienka ciągła
      linia model ciała doskonale czarnego dla $T=1400\K$,
      linia kropkowana reprezentuje model dysku o nachyleniu $i=56^o$, tempie
      akrecji $\dot{M} = 8.2\times10^{-9}\thinspace\Msun\yr^{-1}$
      rozciągającym się od $0.31\AU$ do $705\AU$.}
   }
   \only<2>{
   	\pgfdeclareimage[interpolate=true,width=0.9\textwidth]{mh2}{fig_text}
	\begin{center}
	\pgfuseimage{mh2}
	\end{center}
   }
   \end{column}
\end{columns}

\end{frame}

\begin{frame}
\frametitle{Uwagi}
{\scriptsize
\textbf{Krok połówkowy}
\begin{align}
\mathbf{u}' &= 
\left[
   \left(1 + \gamma\rho'_{\textrm{G}}\Delta t / 2 \right)
   \left(\rho_{\textrm{G}}\mathbf{u} + \mathbf{f}_{\textrm{G}} \Delta t / 2 \right) / \rho'_{\textrm{G}}
   +\gamma \left( \rho_{\textrm{D}}\mathbf{w} + \mathbf{f}_{\textrm{D}}\Delta t / 2 \right) \Delta t / 2
\right] / \Delta, \\
\mathbf{w}' &= 
\left[
   \left(1 + \gamma\rho'_{\textrm{D}}\Delta t / 2 \right)
   \left(\rho_{\textrm{D}}\mathbf{w} + \mathbf{f}_{\textrm{D}} \Delta t / 2 \right) / \rho'_{\textrm{D}}
   +\gamma \left( \rho_{\textrm{G}}\mathbf{u} + \mathbf{f}_{\textrm{G}}\Delta t / 2 \right) \Delta t / 2
\right] / \Delta,
\end{align}
gdzie:
\begin{equation}
\Delta = 1 + \gamma(\rho'_{\textrm{G}} + \rho'_{\textrm{D}})\Delta t / 2.
\end{equation}
\textbf{Krok całkowity}
\begin{align}
\mathbf{u}^{(n+1)} &=
\left[ 
  \left(1 + \gamma \rho_{\textrm{G}}^{(n+1)} \Delta t / 2 \right)\mathbf{\psi}_{\textrm{G}} +
       \left(\gamma \rho_{\textrm{D}}^{(n+1)} \Delta t / 2 \right)\mathbf{\psi}_{\textrm{D}}
\right] / \Delta,  \\
\mathbf{w}^{(n+1)} &= 
\left[ 
  \left(1 + \gamma \rho_{\textrm{D}}^{(n+1)} \Delta t / 2\right)\mathbf{\psi}_{\textrm{D}} +
  \left(  \gamma \rho_{\textrm{G}}^{(n+1)} \Delta t / 2\right) \mathbf{\psi}_{\textrm{G}}
\right] / \Delta,
\end{align}
gdzie:
\begin{align}
\mathbf{\psi}_{\textrm{G}} & = 
\left[
  \rho_{\textrm{G}} \mathbf{u} + \mathbf{f}'_{\textrm{G}}\Delta t - \gamma \rho_{\textrm{G}} \rho_{\textrm{D}} (\mathbf{u} - \mathbf{w}) \Delta t / 2
\right] / {\rho_{\textrm{G}}^{(n+1)}},  \\
\mathbf{\psi}_{\textrm{D}} & = 
\left[
  \rho_{\textrm{D}} \mathbf{w} + \mathbf{f}'_{\textrm{D}}\Delta t - \gamma \rho_{\textrm{G}} \rho_{\textrm{D}} (\mathbf{w} - \mathbf{u}) \Delta t / 2
\right] / {\rho_{\textrm{D}}^{(n+1)}}.
\end{align}
}
\end{frame}

\begin{frame}
\frametitle{Uwagi}
\pgfdeclareimage[interpolate=true,width=0.5\textwidth]{lp}{nosg_vlzd_growth}
\pgfdeclareimage[interpolate=true,width=0.5\textwidth]{pp}{sg_vlzd_growth}
\begin{columns}
   \begin{column}{0.5\textwidth}
      \begin{center}
          \pgfuseimage{lp}
      \end{center}
   \end{column}
   \begin{column}{0.5\textwidth}
      \begin{center}
          \pgfuseimage{pp}
      \end{center}
   \end{column}
\end{columns}
\end{frame}
\end{document}
