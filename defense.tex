\documentclass{beamer}

\mode<presentation>
{
%  \usetheme{Boadilla}
%  \usetheme[blue,compress,numbers,nonav]{Trondheim}
   \usetheme[blue,compress]{Madrid}
  \usefonttheme[onlymath]{serif}
  \setbeamercovered{transparent}
}

\usepackage{tikz}
\usepackage{multimedia}
\usepackage[buttonsize=10.0pt]{animate}
\usepackage{hyperref}

\usepackage[polish]{babel}
\usepackage[utf8]{inputenc}
\usepackage{mathptmx}
\usepackage{helvet}
\usepackage{courier}
\usepackage{graphics}
\usepackage[T1]{fontenc}
\usepackage[absolute,overlay]{textpos}
\setlength{\TPHorizModule}{1mm}
\setlength{\TPVertModule}{1mm}


\selectlanguage{polish}

\title[Powstawanie planetezymali]
{Powstawanie planet w wyniku połączonego działania niestabilności płynowych w dyskach protoplanetarnych}
\author[Kacper Kowalik{}]{Kacper Kowalik}
\institute[CA UMK]{Centrum Astronomii UMK}
\date[15.12.2014]{Obrona rozprawy doktorskiej}

\subject{planet formation}

\begin{document}

\begin{frame}
  \titlepage
\end{frame}

\begin{frame}
%\frametitle{Kontekst astrofizyczny}
Planety formują się w dyskach poprzez zderzenia i zlepianie się cząstek (\alert{hipoteza planetezymali} Safronov, 1969).
\begin{itemize}
\item Od pyłu do planetozymali:
\begin{description}
\uncover<1>{\item[$\mu$m$\rightarrow$cm:] Oddziaływania międzycząsteczkowe (van der Waals, wiązania wodorowe) prowadzą do koagulacji (JD).}
\only<1-2>{\item[cm$\rightarrow$km:] Połączone działanie niestabilności płynowych (MRI, KHI, SI, GI, \ldots).}
\end{description}
\uncover<1>{
\item Od planetozymali do protoplanet:
\begin{description}
\item[km$\rightarrow 10^3$km:] Grawitacja.
\end{description}
\item Od protoplanet do planet:
\begin{description}
\item[Gazowe olbrzymy:] Akrecja gazowych otoczek.
\item[Planety skaliste:] Zderzenia protoplanet.
\end{description}}
\end{itemize}
\end{frame}

\begin{frame}

\pgfdeclareimage[interpolate=true,height=0.4\textheight]{d1}{dullemond1}
\pgfdeclareimage[interpolate=true,height=0.4\textheight]{d2}{dullemond2}
\pgfdeclareimage[interpolate=true,height=0.35\textheight]{mri}{mri}
\pgfdeclareimage[interpolate=true,height=0.36\textheight]{khi}{khi}

%\frametitle{Bariera 1m @ 1 AU}
\begin{block}{Procesy hamujące wzrost cząstek pyłu}
\begin{itemize}
\item Duża względna prędkość cząstek prowadzi do fragmentacji przy zderzeniu (Blum \& Wurm 2008)
\uncover<2->{\item Silne ograniczenie skali czasowej na skutek radialnego dryfu pyłu.}
\uncover<3->{\item Turbulencja (MRI) przeciwdziała sedymentacji.}
\uncover<4->{\item Sedymentacja prowadzi do niestabilności KH.}
\end{itemize}
\end{block}
\begin{center}
\only<1>{\pgfuseimage{d1}}
\only<2>{\pgfuseimage{d2}}
\only<3>{\pgfuseimage{mri} \\
   {\scriptsize Johansen A., \& Klahr H., \emph{ApJ}, \bf{634}}
}
\only<4>{\pgfuseimage{khi} \\
   {\scriptsize Johansen A., Henning Th., \& Klahr H. \emph{ApJ}, \bf{643}}
}  
\end{center}
\end{frame}

\begin{frame}
%\frametitle{Maksima w rozkładzie ciśnienia przechwytują gaz}
\pgfdeclareimage[interpolate=true,height=0.8\textheight]{d3}{dullemond3}
\pgfdeclareimage[interpolate=true,width=0.45\textwidth]{a1}{arm1}
\pgfdeclareimage[interpolate=true,width=0.45\textwidth]{a2}{arm2}

\begin{center}
  \pgfuseimage{a1} \pgfuseimage{a2} \\
  {\scriptsize Rice, W.K.M., G. Lodato, J.E. Pringle, P.J. Armitage, \& I.A. Bonnell, \emph{MNRAS}, \bf{372}}
\end{center}
\end{frame}

\end{document}

\begin{frame}
%\frametitle{Niestabilność strumieniowa}
\pgfdeclareimage[interpolate=true,width=.51\textwidth]{kk}{stream_kk2}
\pgfdeclareimage[interpolate=true,width=.48\textwidth]{aj}{stream_aj}
\begin{columns}
   \begin{column}{0.5\textwidth}
      \begin{center}
         \pgfuseimage{aj}\\
         {\scriptsize Johansen A., \& Youdin A. (2007), \emph{ApJ}, \bf{662}}
      \end{center}
   \end{column}
   \begin{column}{0.5\textwidth}
      \begin{center}
         \movie[poster,externalviewer,label=ccmovie]{\pgfuseimage{kk}}{stream_BA_512.avi}\\
         {\scriptsize Kowalik K., Hanasz M., Wóltański D.,~ \emph{,,New Technologies for Probing the Diversity of Brown Dwarfs and Exoplanets''}, (2009)}
      \end{center}
   \end{column}
\end{columns}
\end{frame}

\begin{frame}
%\frametitle{Current state of PIERNIK art}
\only<1>{
\pgfdeclareimage[interpolate=true,width=0.7\textwidth]{logoHD}{logoHD}
\pgfdeclareimage[interpolate=true,height=2.5cm]{arfur}{arfur}
\pgfdeclareimage[interpolate=true,height=2.5cm]{wolt}{wolt}
\pgfdeclareimage[interpolate=true,height=2.5cm]{szef}{szef}
\pgfdeclareimage[interpolate=true,height=2.5cm]{jd}{jd}
\begin{center}
\pgfuseimage{logoHD}
\end{center}
\begin{center}
\begin{tabular}{cccc}
   \pgfuseimage{jd} & 
   \pgfuseimage{arfur} & 
   \pgfuseimage{szef} &
   \pgfuseimage{wolt} \\ 
   \scriptsize{Joanna Dr\k{a}\.zkowska} &
   \scriptsize{Artur Gawryszczak} & 
   \scriptsize{Micha\l{} Hanasz} &
   \scriptsize{Dominik W\'olta\'nski}
\end{tabular}
\href{http://piernik.astri.umk.pl/}{http://piernik.astri.umk.pl/}
\end{center}
}
\only<2>{
\begin{itemize}
   \item Solver hydrodynamiczny -- Relaxing TVD (Jin \& Xin, 1995; Trac \& Pen, 2003)
   \item Geometria cylindryczna (Skinner \& Ostriker, 2010), wielopłynowość, oddziaływanie między pyłem, a gazem przy użyciu ,,jawnego'' i ,,pół-niejawnego'' schematu (Tilley \& Balsara, 2008).
   \item Samograwitacja (multigrid, multipole, direct FFT)
   \item Efektywna parallelizacja zagadanienia.
\end{itemize}
Warunki początkowe:
{\scriptsize
\begin{enumerate}
   \item MMSN: $\Sigma(R) = 1700\cdot r^{-1.5}$ g cm$^2$, $T(R) = 150 ( R / 1 \textrm{AU} )^{-0.429}$ K.
   \item $[2\div12,0\div2\pi,-0.5\div0.5]$ w $N=(2048,5,256)$
   \item zaburzenie prędkości pyłu $\sim10^{-7} e^{-i(k_x x + k_y y + k_z z)}$ albo biały szum
   \item ,,killing zones'' $\frac{\textrm{d}X}{\textrm{d}t} = -\frac{X-X_0}{\tau}f(r)$, wewnętrzne obcięcie profilu gęstości.
   \item Czas numerycznej relaksacji układu $T_{\textrm{relax}} = 15$ yr.
\end{enumerate}
}
}
\end{frame}

\begin{frame}
\pgfdeclareimage[interpolate=true,height=0.9\textheight]{p1}{plot}
\pgfdeclareimage[interpolate=true,width=0.62\textwidth]{p2}{combined}
\pgfdeclareimage[interpolate=true,width=0.62\textwidth]{p2a}{low}
\pgfdeclareimage[interpolate=true,width=0.62\textwidth]{p2b}{high}
\pgfdeclareimage[interpolate=true,height=0.9\textheight]{p3}{growth}
\begin{center}
%\only<1>{\pgfuseimage{p1}}
\only<1>{
\begin{columns}
   \begin{column}{0.4\textwidth}
     Zróżnicowanie średnicy ziaren pyłu: $10, 50, 100$~cm for $\epsilon=1.0$.
   \end{column}
   \begin{column}{0.6\textwidth}
     \movie[poster,externalviewer,label=ccmovie]{\pgfuseimage{p2}}{combined.avi}
   \end{column}
\end{columns}
\begin{columns}
   \begin{column}{0.4\textwidth}
     Zróżnicowanie $\epsilon$: $0.1, 0.01$ dla $50$~cm ziaren pyłu.
   \end{column}
   \begin{column}{0.6\textwidth}
     \movie[poster,externalviewer,label=ccmovie]{\pgfuseimage{p2a}}{stream_50cm_low.avi}
   \end{column}
\end{columns}
\begin{columns}
   \begin{column}{0.4\textwidth}
     Testy zbieżności
   \end{column}
   \begin{column}{0.6\textwidth}
     \movie[poster,externalviewer,label=ccmovie]{\pgfuseimage{p2b}}{stream_50cm_high.avi}
   \end{column}
\end{columns}
}
%\only<3>{\pgfuseimage{p3}}
\end{center}
\end{frame}

\begin{frame}
%\frametitle{Obecne prace -- Analiza zebranych danych}
\only<1>{
Początkowy etap rozwoju niestabilności strumieniowej (faza liniowa) daje się ,,potraktować'' analitycznie w układzie ,,kostki ścinanej'', poprzez linearyzcję 8 równań PDE opisujących 8 wielkości fizycznych. Otrzymujemy informację na temat stanu układu zapisaną jako superpozycję fal płaskich
\[ \tilde{f}\exp\left[i\left(k_x + k_z - \omega t\right)\right],\textrm{ gdzie }
   \tilde{f}=\left[\tilde{\rho_d},\tilde{w}_x,\tilde{w}_y,\tilde{w}_z,\tilde{\rho_g},\tilde{u}_x,\tilde{u}_y,\tilde{u}_z\right]. \]
\[ \mathbf{A(\omega)} \tilde{f} = \vec{b} \] 
Schemat analizy:
\begin{itemize}
\item Wybieramy z naszej globalnej symulacji obszar na tyle mały, że możemy go traktować jako lokalne przybliżenie dysku (,,łatka'').
\item Dla zbioru stałych określającego naszą ,,łatkę'' znajdujemy rozwiązania zespolone dla równania 8 stopnia ($\det|\mathbf{A(\omega)}|=0 \Rightarrow \omega$).
\item Wyznaczamy wzajemne relacje pomiędzy amplitudami dla ustalonego jednego zaburzenia np. $\rho_d$. ($\tilde{f} = \mathbf{A}^{-1}\vec{b}$) . 
\end{itemize}
}

\only<2>{
\pgfdeclareimage[interpolate=true,width=0.95\textwidth]{a1}{amps}
\begin{center}\pgfuseimage{a1}\end{center}
}
\only<3>{
\pgfdeclareimage[interpolate=true,width=0.95\textwidth]{a1}{fale}
\begin{center}\pgfuseimage{a1}\end{center}
}
\end{frame}

\begin{frame}
%\frametitle{Obecne prace -- pełne symulacje 3D} 
Optymalna liczba procesorów dla symulacji w niskiej rozdzielczości ($2048\times128\times256$) to \alert{256}. Czas obliczeń: 1 miesiąc (\alert{720h}), liczba symulacji: \alert{kilkanaście}.
\pgfdeclareimage[interpolate=true,width=0.5\textwidth]{pl1}{diffusion}
\pgfdeclareimage[interpolate=true,width=0.5\textwidth]{pl2}{dysk}
\begin{columns}
   \begin{column}{0.5\textwidth}
      \pgfuseimage{pl1}
   \end{column}
   \begin{column}{0.5\textwidth}
      \pgfuseimage{pl2}
   \end{column}
\end{columns}
\end{frame}

\frame{
   %\frametitle{Idea}
   \begin{center}
      \includegraphics[width=0.95\textwidth]{idea.png}
      \begin{itemize}
         \item{\includegraphics[height=0.6cm]{piernik_logo_HD.png} }
         \item Algorytm Monte Carlo przy użyciu cząstek reprezentatywnych (Zsom \& Dullemond, 2008)
         \item Prosty model kolizji (Ormel et al., 2007) -- decydujący czynnik $\Delta v$.
      \end{itemize}
   \end{center}
}

\frame{
%\frametitle{Założenia}
   Cząstki:
      \begin{itemize}
         \item są równomiernie rozdystrybuowane po objętości komórek Piernika -- nie ma potrzeby znania ich dokładnego położenia,
         \item mogą się zderzać z innymi cząstkami tylko w obrębie jednej komórki,
         \item mogą przemieszczać się pomiędzy komórkami,
         \item mają określony wektor prędkości,
         \item są poddawane oporowi aerodynamicznemu $\vec{F} = - \frac{1}{t_s}\left(\vec{v_p} - \vec{v_g} \right).$
      \end{itemize}
   Ponadto:
      \begin{itemize}
         \item{Względna prędkość cząstek jest obliczana na podstawie wyrażenia 
               $\Delta v_{ik} = \sqrt{\Delta v_B^2 + \Delta v_T^2 + \left(\vec{v_i} - \vec{v_k}\right)^2}$,\\
               gdzie $\Delta v_B$ wynika z ruchów Browna, \\
               $\Delta v_T$ wynika z ,,podskalowej'' turbulencji, 
               $|\vec{v_i} - \vec{v_k}|$ prędkości cząstek wzbudzone na skutek oddziaływania z gazem}
         \item{Bierzemy pod uwagę dyfuzję pyłu powodowaną podskalową turbulencją}
      \end{itemize}
}

\frame{
   %\frametitle{Adwekcja, dyfuzja}
   \begin{center}
   \includegraphics[height=0.7\textheight]{test2.png}
   \end{center}
}

\begin{frame}
%\frametitle{Widmo energi kinetycznej gazu}
      \begin{center}
      \includegraphics[height=0.7\textheight]{spectrum3d.png}
      \end{center}
\end{frame}

\begin{frame}
%\frametitle{2D}
      \pgfdeclareimage[interpolate=true,width=0.8\textwidth]{turb}{turb.png}
      \begin{center}
      \movie[poster,externalviewer,label=ccmovie]{\pgfuseimage{turb}}{turb.avi}\\
      \end{center}
\end{frame}

\begin{frame}
%\frametitle{3D}
      \begin{center}
      \includegraphics[height=0.7\textheight]{dust3d.png}
      \end{center}
\end{frame}

\begin{frame}
\scriptsize
Problemy:
\begin{enumerate}
  \item \emph{Bardzo silny} wpływ turbulencji na fizyczne wyniki symulacji.
  \item Całość jest \emph{przeraźliwie} wolna.
\end{enumerate}
Wykonana praca:
\begin{itemize}
 \item zrównoleglenie modułu MC pod MPI (04/2011)  (2)
 \item dodanie tracera (05/2011) (1)
 \item przystosowanie modulu turbulencji z kodu Godunov (by G. Kowal) do Piernika (05/2011) (1)
 \item zadaptowanie restartow na potrzeby czastek i turbulencji (06/2011) (1)
 \item dodanie solvera Riemannowskiego (07/2011) (1)
 \item optymilizacja kodu MC (serial) (09/2011) (2)
 \item rozpoczęcie prac nad “portowaniem” MC na GPU (Damian Kaliszan, Artur Trojanowski) (11/2011) (2)
 \item adaptive time-stepping, multiresolution (11/2011) (2)
\end{itemize}
\end{frame}

\begin{frame}
%\frametitle{Dlaczego potrzebujemy nowego solwera -- Wengen4}
Izotermiczny 3D dysk na skraju niestabilności grawitacyjnej.
\begin{columns}[c]
\column{0.5\textwidth}
\pgfdeclareimage[interpolate=true,width=.95\textwidth]{fl}{flash.png}
\pgfuseimage{fl}\\ {\scriptsize Flash (courtesy of A. Gawryszczak)}
\column{0.5\textwidth}
\pgfdeclareimage[interpolate=true,width=.95\textwidth]{rtvd}{rtvd.png}
\pgfdeclareimage[interpolate=true,width=.95\textwidth]{mhm}{mhm.png}
\pgfdeclareimage[interpolate=true,width=.95\textwidth]{rtvd}{rtvd.png}
\alt<2>{\movie[poster,externalviewer,label=ccmovie]{\pgfuseimage{mhm}}{wt4_512x512x104.avi}\\ {\scriptsize Piernik MHM}}
{\pgfuseimage{rtvd}\\ {\scriptsize Piernik RTVD (courtesy of C. Monet)}}
\end{columns}
\end{frame}
\end{document}
